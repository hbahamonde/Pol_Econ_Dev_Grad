% LaTeX Curriculum Vitae Template
%
% Copyright (C) 2004-2009 Jason Blevins <jrblevin@sdf.lonestar.org>
% http://jblevins.org/projects/cv-template/
%
% You may use use this document as a template to create your own CV
% and you may redistribute the source code freely. No attribution is
% required in any resulting documents. I do ask that you please leave
% this notice and the above URL in the source code if you choose to
% redistribute this file.

\documentclass[letterpaper]{article}

\usepackage{hyperref}
\usepackage{geometry}

% Comment the following lines to use the default Computer Modern font
% instead of the Palatino font provided by the mathpazo package.
% Remove the 'osf' bit if you don't like the old style figures.
\usepackage[T1]{fontenc}
\usepackage[sc,osf]{mathpazo}

% Set your name here
\def\name{The Political Economy of Development}

% Replace this with a link to your CV if you like, or set it empty
% (as in \def\footerlink{}) to remove the link in the footer:
\def\footerlink{}
% \href{http://www.hectorbahamonde.com}{www.HectorBahamonde.com}

% The following metadata will show up in the PDF properties
\hypersetup{
  colorlinks = true,
  urlcolor = blue,
  pdfauthor = {\name},
  pdfkeywords = {political science, economic development, methods},
  pdftitle = {\name: Curriculum Vitae},
  pdfsubject = {Curriculum Vitae},
  pdfpagemode = UseNone
}

\geometry{
  body={6.5in, 8.5in},
  left=1.0in,
  top=1.25in
}

% Customize page headers
\pagestyle{myheadings}
\markright{{\tiny \name}}
\thispagestyle{empty}

% Custom section fonts
\usepackage{sectsty}
\sectionfont{\rmfamily\mdseries\Large}
\subsectionfont{\rmfamily\mdseries\itshape\large}

% Other possible font commands include:
% \ttfamily for teletype,
% \sffamily for sans serif,
% \bfseries for bold,
% \scshape for small caps,
% \normalsize, \large, \Large, \LARGE sizes.

% Don't indent paragraphs.
\setlength\parindent{0em}

% Make lists without bullets
\renewenvironment{itemize}{
  \begin{list}{}{
    \setlength{\leftmargin}{1.5em}
  }
}{
  \end{list}
}

\begin{document}

% Place name at left
%{\huge \name}

% Alternatively, print name centered and bold:
\centerline{\huge \bf \name}

\vspace{0.25in}

\begin{minipage}{0.45\linewidth}
  Rutgers University, New Brunswick \\
  Political Science Department \\
  Hickman Hall \\
  New Brunswick, NJ 08901\\
  \\
  \\

\end{minipage}
\hspace{4cm}\begin{minipage}{0.45\linewidth}
  \begin{tabular}{ll}
{\bf Last updated}: \today. \\
 {\bf Download last version} \href{https://github.com/hbahamonde/Pol_Econ_Dev_Grad/raw/master/Pol_Econ_Dev_Syllabus_GRAD.pdf}{here}.
    \\
    \\
    \\
    \\
    \\
    \\
  \end{tabular}
\end{minipage}

\vspace{-5mm}
{\bf Instructor}: H\'ector Bahamonde\\
\texttt{e:}\href{mailto:hector.bahamonde@rutgers.edu}{\texttt{hector.bahamonde@rutgers.edu}}\\
\texttt{w:}\href{http://www.hectorbahamonde.com}{\texttt{www.hectorbahamonde.com}}\\
{\bf Location}: Classroom.\\
{\bf Office Hours}: Make an appointment \href{https://calendly.com/bahamonde/officehours}{\texttt{here}}.

\subsection*{Overview and Objectives}

This {\bf {\color{blue}graduate-level course}} is intended as an introduction to the political economy of institutions and long-run development. The papers will draw from political economy, development economics, economic history, fiscal sociology, institutional economics and some times, applied econometrics. 



\subsection*{Course Learning Objectives}
 
Upon successful completion of this course, you will be able to:

\begin{itemize}
	\item[$\bullet$] Acquire an understanding of the main CPE theories and topics.
	\item[$\bullet$] Use the comparative method and analysis in the political science literature.
	\item[$\bullet$] Consume critically the CPE/Development literature.
	\item[$\bullet$] Produce CPE/Development papers.
\end{itemize}



\subsection*{Requirements}

In this course we will cover the key concepts and theoretical debates in a large sub-field in comparative politics. Students will be expected to complete the required readings each week, attend the lectures, participate in class discussions and take careful notes. When readings the class materials, you should locate the main argument, strengths, weaknesses, and other issues that are of concern. If there are certain questions or points that you think we should specifically examine in class, mark them down and raise them in our class discussions. The course will assume knowledge of 1st year econometrics, and how to run regressions.

\subsection*{Evaluation}


\begin{itemize}
	\item[$\bullet$] {\bf Four three-pages papers answering one of the week's discussion questions}: 15 \%.
	\item[$\bullet$] {\bf Presentation}: 15 \%.
	\item[$\bullet$] {\bf 15-page final paper proposal}: 25 \%.
\end{itemize}


\subsection*{Students with Disabilities}
Students with disabilities who require accommodation should review the following statement from the Office of Disability Services \href{https://ods.rutgers.edu/faculty/syllabus}{\texttt{link}}.



\subsection*{Schedule}

\begin{enumerate}

\item {\bf Institutions, Growth, and the First Economic Revolution}
	\begin{itemize}
		\item[$\bullet$] Lucas, Robert. 2000. ``Some Macroeconomics for the 21st Century,'' \emph{Journal of Economic Perspectives} 14 (Winter): 159-168.
		\item[$\bullet$]Douglass North. 1990. \emph{Institutions, Institutional Change and Economic Performance}. Cambridge: Cambridge University Press. Pages 1-69.
		\item[$\bullet$]Clark, Gregory. 2007. \emph{A Farewell to Alms: A Brief Economic History of the World}. Princeton University Press. Chapters 2-5.
		\item[$\bullet$]Boix, Carles. 2015. \emph{Political Order and Inequality}. Cambridge University Press. Pages 61-65, 85-87, 92- 127.
	\end{itemize}


\item {\bf The Modern Breakthrough}

	\begin{itemize}
		\item[$\bullet$] North, Douglass C. And Barry R. Weingast, 1989. ``Constitutions and Commitment: The Evolution of Institutional Governing Public Choice in Seventeenth-Century England,'' \emph{The Journal of Economic History} 49, (December): 803-832.
		\item[$\bullet$] David Stasavage. 2002. ``Credible Commitment in Early Modern Europe: North and Weingast Revisited,'' \emph{Journal of Law, Economics and Organization} 18(1): 155-186.
		\item[$\bullet$] Daron Acemoglu, Simon Johnson, and James A. Robinson. 2001. ``The Colonial Origins of Comparative Development: An Empirical Investigation,'' \emph{American Economic Review} 91 (December): 1369-1401.
		\item[$\bullet$] E. Glaeser, R. La Porta, and F. Lopez-de-Silanes and A. Shleifer. 2004. ``Do Institutions Cause Growth?'' \emph{Journal of Economic Growth}, September, 2004. Pages 271-303.
		\item[$\bullet$] Clark, Gregory. 2007. \emph{A Farewell to Alms: A Brief Economic History of the World}. Princeton University Press. Chapters 10-13. (B)
		\item[$\bullet$] Mokyr, Joel. 2009. ``The Origins of British Technological Leadership.'' In J. Mokyr, \emph{The Enlightened Economy: An Economic History of Britain, 1700-1850}. Yale University Press. Chapter 6. Pages 99-123.
	\end{itemize}


\item {\bf China vs. Europe}

	\begin{itemize}
		\item[$\bullet$] Jones, Eric. 2003. \emph{The European Miracle}. Cambridge. Third edition. Introduction to second edition, chapters 1-6, and afterword to third edition.
		\item[$\bullet$] Rosenthal, Jean-Laurent, and Roy Bin Wong. 2011. \emph{Before and Beyond Divergence}. Harvard University Press.
	\end{itemize}


\item {\bf Lagging Behind}

	\begin{itemize}
		\item[$\bullet$] Elisa Mariscal and Kenneth L. Sokoloff. 2000. ``Schooling, Suffrage, and the Persistence of Inequality in the Americas, 1800-1945,'' in Stephen Harber, ed. \emph{Political Institutions and Economic Growth in Latin America}. 
		\item[$\bullet$] \emph{Essays in Policy, History, and Political Economy}. Stanford: Hoover Institution Press. Chapter 5, pp. 159-217.
		\item[$\bullet$] Acemoglu, Daron and James A. Robinson. 2000. ``Political Losers as a Barrier to Economic Development,'' American Economic Review 90 (May): 126-130.
		\item[$\bullet$] Clark, Gregory. 2007. \emph{A Farewell to Alms: A Brief Economic History of the World}. Princeton University Press. Chapters 15-17.
		\item[$\bullet$] Robert H. Bates. 1984. \emph{Markets and States in Tropical Africa}. University of California Press.
		\item[$\bullet$] Michael Ross. 2012. \emph{The Oil Curse: How Petroleum Wealth Shapes the Development of Nations}.
		Princeton University Press. Chapter 6.
	\end{itemize}



\item {\bf Catching-Up}

	\begin{itemize}
		\item[$\bullet$] Gerschenkron, Alexander. 1962. \emph{Economic Backwardness in Historical Perspective, a Book of Essays}. Cambridge, Belknap Press of Harvard University Press. Pages 5-30 (``Economic Backwardness in Historical Perspective'') and 353-364 (``The Approach to European Industrialization: A Postscript'').
		\item[$\bullet$] Landes, David S. \emph{The Unbound Prometheus}. New York: Cambridge University Press. Chapters 4 and 5. 
		\item[$\bullet$] Chandler, Alfred D. 1990. \emph{Scale and Scope}. Harvard University Press. Pages 14-36, 47-49, 235-237, 393-395, 593-605.
		\item[$\bullet$] Robert Wade. 1992. ``East Asia's Economic Success: Conflicting Perspectives, Partial Insights, Shaky
		Evidence,'' \emph{World Politics} 44: 270-320.
		\item[$\bullet$] Paul Krugman, ``The Myth of Asia's Miracle,'' \emph{Foreign Affairs}, November/December 1994:63-79.
	\end{itemize}



\item {\bf Ideas, Beliefs and Development}
	\begin{itemize}
		\item[$\bullet$] Max Weber. 2001. \emph{The Protestant Ethic and the Spirit of Capitalism}. Routledge, London. 
		\item[$\bullet$] Robert B. Putnam. 1993. \emph{Making Democracy Work}. Princeton University Press. Chapters 1, 3-6.
		\item[$\bullet$]  Henrich, J., R. Boyd, S. Bowles, H. Gintis, C.Camerer, R. McElreath, E. Fehr, M. Gurven, K. Hill, A. Barr, J. Ensminger, D. Tracer, F. Marlow, J. Patton, M. Alvard, F. Gil-White and N. Henrich. 2005. ``\emph{Economic Man} in Cross-Cultural Perspective: Ethnography and Experiments from 15 Small-Scale Societies,'' \emph{Behavioral and Brain Sciences} 28: 795-855.
		\item[$\bullet$]  Fehr, Ernst, Karla Hoff and Mayuresh Kshetramade. 2008. ``Spite and Development,'' \emph{American Economic Review} 98 (2): 494-499.
	\end{itemize}


\item {\bf Democratic Capitalism}

	\begin{itemize}
		\item[$\bullet$] Piketty, Thomas. 2014. \emph{Capital in the Twenty-First Century}. Harvard/Belknap. Chapters 1, 3-10.
		\item[$\bullet$] David N. Weil. 2015. ``Capital and Wealth in the 21st Century.'' \emph{National Bureau of Economic Research} WP no. 20919.
	\end{itemize}


\item {\bf Keynesian Settlement? (1)}

	\begin{itemize}
		\item[$\bullet$] Adam Przeworski and Michael Wallerstein. 1986. ``Democratic Capitalism at the Crossroads,'' in A. Przeworski. \emph{Capitalism and Social Democracy}. Cambridge: Cambridge University Press. Pages 205-217.
		\item[$\bullet$] Peter Hall, ed. 1989. \emph{The Political Power of Economic Ideas: Keynesianism across Nations}. Princeton University Press. Chapters 1, 3-4, 5, 9-10, 12, 14.
		\item[$\bullet$] Alberto Alesina et al. 1997. \emph{Political Cycles and the Macroeconomy}. Cambridge, Mass: The MIT Press. Pages 1-110.
	\end{itemize}

\item {\bf Keynesian Settlement? (2)}

	\begin{itemize}
		\item[$\bullet$] Alberto Alesina et al. 1997. \emph{Political Cycles and the Macroeconomy}. Cambridge, Mass: The MIT Press. Pages 141-209. 
		\item[$\bullet$] Alberto Alesina and Lawrence Summers. 1993. ``Central Bank Independence and Macroeconomic Performance,'' \emph{Journal of Money, Credit and Banking} 25: 151-162.
		\item[$\bullet$] R.M. Alvarez, G. Garrett and P. Lange. 1991. ``Government Partisanship, Labor Organization and Macroeconomic Performance, 1967-1984,'' \emph{American Political Science Review} 85: 539-556.
		\item[$\bullet$] Kenneth Scheve and David Stasavage. 2009. ``Institutions, Partisanship, and Inequality in the Long Run,'' \emph{World Politics}, 61 (April): 215-253.
		\item[$\bullet$] Baccaro, Lucio, and Jonas Pontusson. 2015. ``Rethinking Comparative Political Economy: Growth Models and Distributive Dynamics.''
	\end{itemize}

\item {\bf Welfare States}

	\begin{itemize}
		\item[$\bullet$] Kenneth Scheve and David Stasavage. 2015. \emph{Taxing the Rich: Fairness and Fiscal Sacrifice Over Two Centuries}. Book manuscript. Chapters TBA.
		\item[$\bullet$] Peter Lindert. 2004. \emph{Growing Public. Social Spending and Economic Growth Since the Eighteenth Century}. New York: Cambridge University Press. Chapters 10-11.
		\item[$\bullet$] Fochesato, Mattia, and Samuel Bowles. 2014. ``Nordic Exceptionalism? Social Democratic Egalitarianism in World-Historic Perspective.'' \emph{Journal of Public Economics}.
		\item[$\bullet$] Adam Przeworski et al. 2000. \emph{Democracy and Development}. Cambridge University Press. Chapter 5. 
	\end{itemize}

\item {\bf Democratic Capitalism Revisited}

	\begin{itemize}
		\item[$\bullet$] Adam Przeworski and Michael Wallerstein. 1986. ``Material Interests, Class Compromise, and the State,'' in A. Przeworski. \emph{Capitalism and Social Democracy}. Cambridge: Cambridge University Press. Pages 171-203.
		\item[$\bullet$] Claudia Goldin, Lawrence F Katz. 2007. ``The race between education and technology: the evolution of US educational wage differentials, 1890 to 2005.'' \emph{NBER} WP no. 12984.
		\item[$\bullet$] Autor, D. 2010. ``The polarization of job opportunities in the US labor market: Implications for employment and earnings.'' \emph{Center for American Progress and The Hamilton Project}.
		\item[$\bullet$] Boix, Carles. 2015. ``Democratic Capitalism at the Crossroads?'' Book manuscript. 
	\end{itemize}


\item {\bf Open Political Economies}

	\begin{itemize}
		\item[$\bullet$] Przeworski, Adam and Covadonga Meseguer. 2006. ``Globalization and Democracy.'' In Pranab Bardhan, Samuel Bowles and Michael Wallerstein, eds. 2006. \emph{Globalization and Egalitarian Redistribution}. Princeton University and Russell Sage Foundation. Chapter 7.
		\item[$\bullet$] Boix, Carles. 2006. ``Between Protectionism and Compensation: The Political Economy of Trade.'' In Pranab Bardhan, Samuel Bowles and Michael Wallerstein, eds. \emph{Globalization and Egalitarian Redistribution}. Princeton University and Russell Sage Foundation.
		\item[$\bullet$] Odd Aukrust. ``Inflation in the Open Economy: A Norwegian Model,'' in L. Krause and W. Salant, eds., \emph{Worldwide Inflation}. Washington, D.C.: Brookings. Pages 109-126.
		\item[$\bullet$] Wibbels, E. and Ahlquist, J. S. (2011), ``Development, Trade, and Social Insurance''. \emph{International Studies Quarterly}, 55: 125-149.
		\item[$\bullet$] Jeffrey G. Williamson. 1998. ``Globalization and the labor market: using history to inform policy.'' In Philippe Aghion and Jeffrey G. Williamson. \emph{Growth, Inequality and Globalization}. New York: Cambridge University Press. Pages 105-200. 
	\end{itemize}



\end{enumerate}




  


  










%\bibliographystyle{plainnat}
%\bibliography{/Users/hectorbahamonde/RU/Bibliografia_PoliSci/Bahamonde_BibTex2013}

\end{document}